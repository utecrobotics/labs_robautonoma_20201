\documentclass[a4paper,11pt]{robotlabs}

% Makes tables look nicer
\usepackage{booktabs}
\renewcommand{\arraystretch}{1.2}


% To change size of verbatim
\usepackage{verbatimbox}
\include{commands}

\definecolor{lbcolor}{rgb}{0.9,0.9,0.9}
\definecolor{colorkey}{rgb}{0,0,1}
\definecolor{colorcomment}{rgb}{0, 0.6, 0.3}
\definecolor{colorstring}{rgb}{0.627,0.126,0.941}
\definecolor{colornumber}{rgb}{0.205, 0.142, 0.73}
\definecolor{forestgreen}{RGB}{34,139,34}
\definecolor{orangered}{RGB}{239,134,64}
\definecolor{darkblue}{rgb}{0.0,0.0,0.6}
\definecolor{gray}{rgb}{0.4,0.4,0.4}
\definecolor{orangered2}{RGB}{128,0,0}

\newtcblisting{listingshell}
{
  listing only,
  colframe=white,
  listing options={
    basicstyle=\small\ttfamily,
    % columns=full,
    columns=fullflexible,
    literate=
      {\$}{{\textcolor{blue}{\$ }}}{1}
      {\~} {$\sim$}{1}
  },
}

\newtcblisting{listingnodollar}
{
  listing only,
  colframe=white,
  listing options={
    basicstyle=\small\ttfamily,
    % columns=full,
    columns=fullflexible,
    literate=
    %   {\$}{{\textcolor{blue}{\$ }}}{1}
      {\~} {$\sim$}{1}
  },
}

\newtcblisting{listingpython}[1]{%
  listing only,
  colframe=gray!20!white,
  coltitle=black,
  title=\footnotesize{#1},
  listing options={
    language=python,
    upquote=true,
    basicstyle=\small\ttfamily,
    columns=fullflexible,
    extendedchars=false,
    breaklines=true,
    showtabs=false,
    showspaces=false,
    showstringspaces=false,
    identifierstyle=\ttfamily,
    keywordstyle=\color{colorkey},
    commentstyle=\color{colorcomment},
    stringstyle=\color{colorstring},
    numberstyle=\color{colornumber},
    breaklines=True,
    floatplacement=H
  },
}


\newtcblisting{listingxml}[1]{%
  listing only,
  colframe=gray!20!white,
  coltitle=black,
  title=\footnotesize{#1},
  listing options={
    language=XML,
    basicstyle=\small\ttfamily,
    columns=fullflexible,
    extendedchars=false,
    breaklines=true,
    showtabs=false,
    showspaces=false,
    showstringspaces=false,
    identifierstyle=\ttfamily,
    commentstyle=\color{gray},
    keywordstyle=\color{orangered2},
    stringstyle=\color{black},
    tagstyle=\color{colorkey},
    morekeywords={attribute,xmlns,version,type,release,package,name,
      description,maintainer,license,buildtool_depend,build_depend,
      run_depend, length, radius, rpy, xyz, rgba, link, joint,
      effort, velocity, lower, upper},
    otherkeywords={attribute=, xmlns=},
  },
}


\newtcblisting{listingpython2}{%
  listing only,
  colframe=gray!20!white,
  coltitle=black,
  listing options={
    language=python,
    upquote=true,
    basicstyle=\small\ttfamily,
    columns=fullflexible,
    extendedchars=false,
    breaklines=true,
    showtabs=false,
    showspaces=false,
    showstringspaces=false,
    identifierstyle=\ttfamily,
    keywordstyle=\color{colorkey},
    commentstyle=\color{colorcomment},
    stringstyle=\color{colorstring},
    numberstyle=\color{colornumber},
    morekeywords={self}
    breaklines=True,
    floatplacement=H
  },
}

\title{
  \textbf{Laboratorio 5 \\ Sensores en URDF y SLAM en 3D}
  \date{}
}

\begin{document}

\course{Robótica Autónoma 2020-1}
\period{2020-1}
\labnumber{5}
\labname{Sensores en URDF y SLAM en 3D}

% Carátula
% -----------
\thispagestyle{empty}
\begin{center}
\textbf{{\huge Universidad de Ingenier\'ia y Tecnolog\'ia}\\ [.5cm]
 {\LARGE Departamento de Ingenier\'ia  Mecatr\'onica}\\[3cm]
}
{\includegraphics[width=6cm]{images/utec}}\\[3cm]
\end{center}

\begin{center}
  {\LARGE \textbf{Robótica Autónoma}}\\[0.8cm]
  {\Large \textbf{Profesor: Oscar Ramos Ponce}}\\[3.0cm]
  {\Large \textbf{Laboratorio 5}}\\[0.5cm]
  {\Large \textbf{Sensores en URDF y SLAM en 3D}}\\[4.8cm] %[2.0cm]
  {\Large \textbf{Lima - Perú}} \\[0.5cm]
  {\LARGE \textbf{2020-1}}
\end{center}

\newpage
\maketitle
\thispagestyle{fancyplain}

%\tableofcontents
%\thispagestyle{empty}

\section{Objetivos}

\begin{itemize}
\item Analizar los componentes que definen un sensor en un modelo URDF para
  visualización y para simulación con Gazebo
\item Incorporar un sensor Kinect sobre un robot Turtlebot3 en el modelo URDF
  para su simulación.
\item Obtener el mapa tridimensional de un entorno usando SLAM a través del
  paquete llamado \textit{RTAB-Map}.
\end{itemize}

\section{Equipo y Materiales}

\begin{table*}[h]
  \centering
  \begin{tabular}{cc}
    \toprule
    \textbf{Descripción} & \textbf{Cantidad} \\
    \midrule
    Computadora con conexi\'on a internet & 1 \\
    \bottomrule
    \end{tabular}
  \end{table*}

\section{Aspectos Generales}

Para la presentación del reporte  del laboratorio se debe tener en cuenta lo
siguiente.
\begin{itemize}
\item El reporte debe ser desarrollado utilizando cualquier editor de texto,
  pero debe ser presentado en formato pdf a través del buzón de
  Canvas.
\item En el reporte se debe indicar claramente la sección para la cual se está
  brindando la respuesta a la correspondiente actividad.
\item La presentación del reporte se realizará hasta el término de la sesión de
  laboratorio.
\item Se trabajará en grupos de 2 y 3 personas y solo es necesario que un
  integrante de cada grupo presente el reporte, indicando el nombre de los
  integrantes del grupo.
\end{itemize}
Para trabajar en este laboratorio, es necesario actualizar el repositorio de
git clonado en el laboratorio anterior. Para ello, se debe ir a la carpeta
\texttt{autlabs} y se debe ejecutar \texttt{git pull}:
\begin{listingshell}
$ cd ~/lab_ws/src/autlabs
$ git pull
\end{listingshell}
\noindent Una vez realizado esto, dentro de la carpeta \texttt{autlabs} se debe
tener una nueva carpeta llamada \texttt{autlab5}, que es un paquete de ROS y es
donde se trabajará este laboratorio.


\section{Sensores del Robot Turtlebot3}

En esta sección se busca primero comprender cómo se utiliza los sensores en
URDF, para posteriormente poder hacer modificaciones en los sensores que tiene
por defecto el robot Turtlebot 3. En particular, se buscará quitar la cámara
RGB y el LiDAR, y añadir una cámara RGBD (Kinect).

\subsection{Sensores en URDF y Gazebo (8 puntos) }

Como se realizó en laboratorios anteriores, se puede correr usando
\texttt{roslaunch} el archivo de lanzamiento \texttt{turtlebot3\_world.launch}
que se encuentra en el paquete \texttt{autlab5}. De hacerlo se verá el mismo
entorno que se observó en el laboratorio anterior. Al abrir este archivo de
lanzamiento con un editor de texto (configurando para que sea mostrado como de
tipo XML, de preferencia), se puede ver que internamente llama al archivo
\texttt{urdf/turtlebot3.urdf.xacro}. Abrir este último archivo con un editor de
texto para su análisis.


\paragraph{Actividades.} Las preguntas de esta actividad se basan en un
análisis del modelo URDF del robot, que se encuentra en el archivo
\texttt{autlab5/urdf/turtlebot3.urdf.xacro}.

\begin{enumerate}
\item (1 punto) Realizar una descripción de los eslabones (\textit{links}) y
  articulaciones (\textit{joints}) que tiene este robot, con respecto a sus
  nombres y cómo están unidos (cuáles se unen con cuáles). No incluir los
  elementos asociados a los sensores.
  
\item (1 punto) Identificar los eslabones y articulaciones asociados a los
  sensores del robot: IMU, LiDAR (scan) y cámara RGB. Para cada uno de estos
  sensores indicar cuál es su eslabón asociado, cuál es su articulación, cuál
  es su posición y orientación (escoger un sistema de referencia adecuado), y
  si tiene algún elemento que permita su visualización (malla, por ejemplo).

\item (0.5 puntos) Ejecutar el archivo \texttt{turtlebot3\_world.launch} y verificar
  los tópicos. Indicar el contenido del tópico \texttt{/tf}, indicando qué
  sistemas de referencia relaciona y por qué se obtiene los valores mostrados
  en el tópico. Igualmente, ejecutar \texttt{rqt\_tf\_tree} y mostrar lo
  obtenido.

\item (0.5 puntos) Abrir rviz y añadir un componente de tipo
  RobotModel. Indicar si se puede visualizar el robot y por qué sí o por qué
  no.

\item (1 punto) En otro terminal ejecutar el nodo
  \texttt{robot\_state\_publisher}, que está contenido en un paquete que tiene
  ese mismo nombre. Ejecutar nuevamente \texttt{rqt\_tf\_tree}, mostrando su
  contenido. Igualmente, mostrar el contenido del tópico \texttt{/tf} y del
  tópico \texttt{/tf\_static}. Abrir nuevamente RViz e indicar si se puede
  visualizar el robot.

\item (0.5 puntos) A partir de lo observado, ¿qué hace
  \texttt{robot\_state\_publisher}?
  
\item (1 punto) Desde el URDF que se está utilizando para el robot, se hace un
  llamado a \texttt{turtlebot3.gazebo.xacro}. Abrir este archivo y responder:
  ¿qué parámetros se usa dentro del tag \texttt{gazebo}? y ¿qué representan?

\item (0.5 puntos) ¿Qué elementos en este archivo de simulación hacen uso de un
  plugin, y cómo se llaman estos plugins?
  
\item (0.5 puntos) ¿Qué es un plugin de Gazebo y para qué se utiliza?

\item (1 punto) Al inicio del archivo \texttt{turtlebot3.gazebo.xacro} hay 3
  argumentos cuyos valores son falsos. Cambiar a verdadero el valor
  correspondiente al láser y correr nuevamente la simulación de Gazebo. ¿Qué se
  observa?. Luego regresar dicho valor a falso y poner a verdadero el valor
  correspondiente a la cámara y correr nuevamente la simulación de Gazebo. ¿Qué
  son las líneas que se observa? Responder y adjuntar los resultados de cada
  parte.

\item (0.5 puntos) En la simulación de Gazebo se observa un LiDAR sobre el Turtlebot3, pero,
  a pesar de que hay una cámara, no se observa la cámara sobre el robot. ¿Por
  qué sucede esto?
  
\end{enumerate}


\subsection{Incorporación de Sensor Kinect al Turtlebot 3 (4 puntos)}

Una vez que se ha analizado los sensores del robot en el URDF y en el modelo de
Gazebo, en esta sección se quitará el sensor LiDAR y la cámara RGB del
Turtlebot3 y se añadirá un Kinect (sensor RGB-D). Con este fin, primero copiar
el archivo \texttt{turtlebot3\_world.launch} a
\texttt{turtlebot3\_kinect\_world.launch}. Igualmente, copiar el archivo
\texttt{turtlebot3.urdf.xacro} a
\texttt{turtlebot3\_kinect.urdf.xacro}. Finalmente, copiar el archivo
\texttt{turtlebot3.gazebo.xacro} a
\texttt{turtlebot3\_kinect.gazebo.xacro}. Notar que en todos los casos es
recomendable copiar y no renombrar los archivos.

Una vez que se realizó la copia de cada uno de los archivos antes mencionados,
modificar el archivo de lanzamiento \texttt{turtlebot3\_kinect\_world.launch}
para que haga un llamado al modelo del robot que se encuentra en
\texttt{turtlebot3\_kinect.urdf.xacro}. Igualmente, este último archivo deba
hacer un llamado al archivo \texttt{turtlebot3\_kinect.gazebo.xacro}. Correr el
archivo de tipo \texttt{launch} para verificar que hasta el momento todo esté
funcionando. 

\paragraph{Actividades.}

\begin{enumerate}
\item (1 punto) Quitar el LiDAR y la cámara RGB del modelo URDF (y de la
  simulación de Gazebo). Indicar dónde se realizó la modificación, y mostrar el
  modelo en Gazebo sin el LiDAR. Nota: es importante quitar todos los
  componentes de la cámara (\textit{links} y \textit{joints} asociados,
  inclusive) para evitar conflictos con el Kinect.
  
\item (1 punto) El archivo \texttt{kinect\_gazebo.txt} contiene los tags de
  Gazebo que debe añadirse al archivo de Gazebo para poder simular el
  robot. Por otro lado, el archivo \texttt{kinect.urdf.xacro} contiene los
  datos relativos al Kinect (\textit{links}, \textit{joints}, etc.). Desde
  \texttt{turtlebot3\_kinect.urdf.xacro}, incluir este archivo
  (\texttt{kinect.urdf.xacro}) usando un tag, e indicar que el \textit{parent}
  de este sensor es \texttt{base\_link}: 
\begin{listingshell}
<xacro:include filename="$(find autlab5)/urdf/kinect.urdf.xacro"/>
<xacro:sensor_kinect parent="base_link"/>
\end{listingshell}
%$
Ejecutar el archivo de lanzamiento y mostrar el robot con el Kinect sobre el
robot.

\item (0.5 puntos) Modificar adecuadamente los parámetros que sean necesarios
  para que el Kinect aparezca en un lugar adecuado sobre el robot (no demasiado
  arriba ni demasiado abajo). Mostrar la simulación con el sensor Kinect en un
  lugar adecuado, así como el cambio realizado.
  
\item (0.5 puntos) ¿Qué plugins utiliza el sensor Kinect para la simulación de
  Gazebo?

\item (1 punto) Utilizando RViz, visualizar el robot y la nube de puntos
  generada por el sensor Kinect. Adjuntar el resultado mostrando la nube de
  puntos de al menos dos maneras diferentes (Puntos, cuadrados de distinto
  tamaño, etc.)

  
\end{enumerate}


\section{SLAM en 3D}

Utilizando el sensor Kinect es posible realizar un mapeo tridimensional del
entorno del robot usando SLAM. En esta sección se utilizará una técnica de
RGB-D graph slam llamada RTAB-map (\textit{Real-Time Appearance-Based
  Mapping}) que hace uso de la información que provee el sensor Kinect.

\subsection{Usando el Robot Turtlebot3 con Kinect (4 pts)}

Para poder realizar el escaneo tridimensional, es necesario primero instalar el
paquete de ROS llamado rtabmap-ros de la siguiente manera
\begin{listingshell}
$ sudo apt-get install ros-version-rtabmap-ros
\end{listingshell}
%$
\noindent donde versión puede ser \textit{kinetic} o \textit{melodic}, según
sea el caso. Luego, iniciar la simulación del robot Turtlebot3 con Kinect de la
sección anterior, asegurándose de que se esté publicando las relaciones entre
los diversos sistemas de referencia de las partes del robot (en los tópicos
\texttt{/tf} y \texttt{/tf\_static}). Para iniciar el mapeo y localización
simultáneos en 3D es necesario correr lo siguiente desde otro terminal:
\begin{listingshell}
$ roslaunch rtabmap_ros rtabmap.launch rtabmap_args:="--delete_db_on_start"
  visual_odometry:=false odom_topic:=/odom depth_topic:=/camera/depth/image_raw
  rgb_topic:=/camera/rgb/image_raw rviz:=true rtabmapviz:=false
\end{listingshell}
%$
\noindent Nota: todo el comando anterior debe estar en una sola línea. Este
comando ejecuta RTAB map en la simulación de Gazebo del robot. En RViz puede
ser conveniente desactivar la visualización de los sistemas de referencia
(\textit{TF}) y puede ser útil añadir un elemento \textit{RobotModel} para
visualizar al robot.

Para crear el mapa, es necesario mover al robot, de preferencia no tan
rápido. El movimiento del robot se realizará mediante comandos remotos
introducidos a través del teclado, con el nodo utilizado en laboratorios
anteriores (\texttt{turtlebot3\_teleop}).  Una vez que se termina el mapeo se
puede cerrar los nodos ejecutados (ya que todo este proceso es
computacionalmente costoso). El mapa generado se almacena automáticamente en
\texttt{$\sim$/.ros/rtabmap.db}. Es importante notar que si se ejecuta
nuevamente el comando de mapeo, se sobre escribirá este archivo (por las
opciones que se está pasando a \texttt{rtabmap}).

\paragraph{Actividades.}
\begin{enumerate}
\item (1 punto)  ¿Cómo funciona RTAB-map? Brindar una descripción.
\item (1 punto) Realizar un mapa 3D del entorno en el cual se encuentra el
  robot. Mover el robot usando el teclado como en laboratorios
  anteriores. Adjuntar una captura de pantalla del mapa final obtenido en
  RViz. Nota: una vez que se termina de realizar el mapa, es recomendable
  cerrar gazebo y el programa que realiza el mapeo.
\item (1 punto) Una vez que se tiene el archivo que contiene el mapa
  (almacenado por defecto en \texttt{$\sim$/.ros/rtabmap.db}), abrir el archivo
  de la siguiente manera:
\begin{listingshell}
  $ rtabmap-databaseViewer rtabmap.db
\end{listingshell}
%$
donde se tiene que proporcionar la ruta adecuada al archivo. Explorar el
visualizador y mostrar algunas vistas diferentes del mapa generado (por
ejemplo, las celdas de ocupación (\textit{occupancy grids}) y la vista gráfica
(\textit{graph view}).

\item (0.5 puntos) ¿Qué tipos de movimiento ayudan más al mapeo? Es decir, en
  ¿en qué casos sería recomendable girar o avanzar/retroceder?

\item (0.5 puntos) ¿Qué significa cada uno de los argumentos del comando de
  \texttt{rtabmap\_ros} que se usa para iniciar el mapeo?

\end{enumerate}


\subsection{Usando otro Robot (3 pts)}

Al igual que en el laboratorio anterior, utilizar el modelo de algún robot
móvil (terrestre o aéreo) disponible en URDF para ROS, el cual no disponga de
un sensor RGB-D. Añadir un sensor RBG-D (Kinect, en este caso) al robot
escogido y realizar el mapeo de algún entorno (puede ser el mismo entorno o un
entorno diferente de Gazebo). Se debe mostrar algunas indicaciones de lo
realizado y, de ser posible, el mapa obtenido.


\section{Conclusiones (1 pt)}

Mencionar algunas conclusiones relevantes al desarrollo de este laboratorio.


  

\end{document}

