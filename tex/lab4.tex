\documentclass[a4paper,11pt]{robotlabs}

% Makes tables look nicer
\usepackage{booktabs}
\renewcommand{\arraystretch}{1.2}


% To change size of verbatim
\usepackage{verbatimbox}
\include{commands}

\definecolor{lbcolor}{rgb}{0.9,0.9,0.9}
\definecolor{colorkey}{rgb}{0,0,1}
\definecolor{colorcomment}{rgb}{0, 0.6, 0.3}
\definecolor{colorstring}{rgb}{0.627,0.126,0.941}
\definecolor{colornumber}{rgb}{0.205, 0.142, 0.73}
\definecolor{forestgreen}{RGB}{34,139,34}
\definecolor{orangered}{RGB}{239,134,64}
\definecolor{darkblue}{rgb}{0.0,0.0,0.6}
\definecolor{gray}{rgb}{0.4,0.4,0.4}
\definecolor{orangered2}{RGB}{128,0,0}

\newtcblisting{listingshell}
{
  listing only,
  colframe=white,
  listing options={
    basicstyle=\small\ttfamily,
    % columns=full,
    columns=fullflexible,
    literate=
      {\$}{{\textcolor{blue}{\$ }}}{1}
      {\~} {$\sim$}{1}
  },
}

\newtcblisting{listingnodollar}
{
  listing only,
  colframe=white,
  listing options={
    basicstyle=\small\ttfamily,
    % columns=full,
    columns=fullflexible,
    literate=
    %   {\$}{{\textcolor{blue}{\$ }}}{1}
      {\~} {$\sim$}{1}
  },
}

\newtcblisting{listingpython}[1]{%
  listing only,
  colframe=gray!20!white,
  coltitle=black,
  title=\footnotesize{#1},
  listing options={
    language=python,
    upquote=true,
    basicstyle=\small\ttfamily,
    columns=fullflexible,
    extendedchars=false,
    breaklines=true,
    showtabs=false,
    showspaces=false,
    showstringspaces=false,
    identifierstyle=\ttfamily,
    keywordstyle=\color{colorkey},
    commentstyle=\color{colorcomment},
    stringstyle=\color{colorstring},
    numberstyle=\color{colornumber},
    breaklines=True,
    floatplacement=H
  },
}


\newtcblisting{listingxml}[1]{%
  listing only,
  colframe=gray!20!white,
  coltitle=black,
  title=\footnotesize{#1},
  listing options={
    language=XML,
    basicstyle=\small\ttfamily,
    columns=fullflexible,
    extendedchars=false,
    breaklines=true,
    showtabs=false,
    showspaces=false,
    showstringspaces=false,
    identifierstyle=\ttfamily,
    commentstyle=\color{gray},
    keywordstyle=\color{orangered2},
    stringstyle=\color{black},
    tagstyle=\color{colorkey},
    morekeywords={attribute,xmlns,version,type,release,package,name,
      description,maintainer,license,buildtool_depend,build_depend,
      run_depend, length, radius, rpy, xyz, rgba, link, joint,
      effort, velocity, lower, upper},
    otherkeywords={attribute=, xmlns=},
  },
}


\newtcblisting{listingpython2}{%
  listing only,
  colframe=gray!20!white,
  coltitle=black,
  listing options={
    language=python,
    upquote=true,
    basicstyle=\small\ttfamily,
    columns=fullflexible,
    extendedchars=false,
    breaklines=true,
    showtabs=false,
    showspaces=false,
    showstringspaces=false,
    identifierstyle=\ttfamily,
    keywordstyle=\color{colorkey},
    commentstyle=\color{colorcomment},
    stringstyle=\color{colorstring},
    numberstyle=\color{colornumber},
    morekeywords={self}
    breaklines=True,
    floatplacement=H
  },
}

\title{
  \textbf{Laboratorio 4 \\ SLAM y Navegación de un Robot Móvil}
  \date{}
}

\begin{document}

\course{Robótica Autónoma 2020-1}
\period{2020-1}
\labnumber{4}
\labname{SLAM y Navegación de un Robot Móvil}

% Carátula
% -----------
\thispagestyle{empty}
\begin{center}
\textbf{{\huge Universidad de Ingenier\'ia y Tecnolog\'ia}\\ [.5cm]
 {\LARGE Departamento de Ingenier\'ia  Mecatr\'onica}\\[3cm]
}
{\includegraphics[width=6cm]{images/utec}}\\[3cm]
\end{center}

\begin{center}
  {\LARGE \textbf{Robótica Autónoma}}\\[0.8cm]
  {\Large \textbf{Profesor: Oscar Ramos Ponce}}\\[3.0cm]
  {\Large \textbf{Laboratorio 4}}\\[0.5cm]
  {\Large \textbf{SLAM y Navegación de un Robot Móvil}}\\[4.8cm] %[2.0cm]
  {\Large \textbf{Lima - Perú}} \\[0.5cm]
  {\LARGE \textbf{2020-1}}
\end{center}

\newpage
\maketitle
\thispagestyle{fancyplain}

\section{Objetivos}

\begin{itemize}
\item Obtener el mapa bidimensional de un entorno usando SLAM a través del
  paquete llamado \textit{gmapping}.
\item Generar movimiento autónomo para un robot móvil para llegar a una
  posición y orientación deseados usando el paquete \textit{amcl}.
\end{itemize}

\section{Equipo y Materiales}

\begin{table*}[h]
  \centering
  \begin{tabular}{cc}
    \toprule
    \textbf{Descripción} & \textbf{Cantidad} \\
    \midrule
    Computadora con conexi\'on a internet & 1 \\
    \bottomrule
    \end{tabular}
  \end{table*}

\section{Aspectos Generales}

Para la presentación del reporte  del laboratorio se debe tener en cuenta lo
siguiente.
\begin{itemize}
\item El reporte debe ser desarrollado utilizando cualquier editor de texto,
  pero debe ser presentado en formato pdf a través del buzón de
  Canvas.
\item En el reporte se debe indicar claramente la sección para la cual se está
  brindando la respuesta a la correspondiente actividad.
\item La presentación del reporte se realizará hasta el término de la sesión de
  laboratorio.
\item Se trabajará en grupos de 2 y 3 personas y solo es necesario que un
  integrante de cada grupo presente el reporte, indicando el nombre de los
  integrantes del grupo.
\end{itemize}
Para trabajar en este laboratorio, es necesario actualizar el repositorio de
git clonado en el laboratorio anterior. Para ello, se debe ir a la carpeta
\texttt{autlabs} y se debe ejecutar \texttt{git pull}:
\begin{listingshell}
$ cd ~/lab_ws/src/autlabs
$ git pull
\end{listingshell}
\noindent Una vez realizado esto, dentro de la carpeta \texttt{autlabs} se debe
tener una nueva carpeta llamada \texttt{autlab4}, que es un paquete de ROS y es
donde se trabajará este laboratorio.

\noindent \textit{Nota}: dado que en este laboratorio se utilizará el modelo
waffle\_pi del robot \textit{Turtlebot3}, es necesario definir una variable de
entorno llamada \texttt{TURTLEBOT3\_MODEL} con valor igual a
\texttt{waffle\_pi}. Esta variable fue añadida a \texttt{.bashrc} en el
laboratorio 1, así que no debería haber problemas. En caso que se esté
trabajando en una nueva máquina (o nuevo entorno de ROSDS) es necesario
configurar esta variable como se realizó en el laboratorio 1.


\section{Localización y Mapeo Simultáneos Obstáculos (7 pts)}

Para obtener un mapa en 2 dimensiones se utilizará los datos obtenidos a través
de un LiDAR que funciona en el plano XY y que publica los rangos medidos en el
tópico llamado \texttt{/scan}. El paquete denominado \texttt{gmapping} utiliza
los datos de profundidad en un plano horizontal para realizar el mapa del
entorno mientras se localiza simultáneamente al robot móvil dentro del mapa
generado. Este paquete se encuentra configurado para leer los datos de este
tópico donde publica el LiDAR.

Como primer paso, se requiere instalar los siguientes paquetes en la
computadora (o verificar que se encuentran instalados): \texttt{map\_server},
\texttt{move\_base}, \texttt{amcl}, \texttt{dwa\_local\_planner} y
\texttt{gmapping}. Pueden ser instalados usando \texttt{apt-get install},
anteponiendo al nombre del paquete \texttt{ros-kinetic} o \texttt{ros-melodic},
según sea el caso, y reemplazando los guiones bajos por guiones. Por ejemplo,
para instalar el paquete \texttt{move\_base} con ROS Melodic se tendría el
comando \texttt{sudo apt-get install ros-melodic-move-base}.

Una vez que los paquetes necesarios hayan sido instalados, se debe lanzar el
entorno de Gazebo donde se tiene un robot turtlebot3 simulado usando:
\begin{listingshell}
$ roslaunch autlab4 turtlebot3_world1.launch
\end{listingshell}
%$
\noindent Luego, se debe ejecutar los siguientes dos comandos, cada uno desde
un terminal diferente:
\begin{listingshell}
$ roslaunch autlab4 show_map.launch 
$ roslaunch autlab4 turtlebot3_slam_mapping.launch 
\end{listingshell}
\noindent Para crear el mapa, es necesario mover al robot. Por facilidad, y
dado que en la práctica es bastante común, se comandará remotamente el robot de
manera manual para generar su movimiento. Esto se puede realizar a través del
teclado usando el nodo utilizado en el laboratorio 1
(\texttt{turtlebot3\_teleop}).


\paragraph{Actividades.}
\begin{enumerate}
\item (2 pts)  Hacer un breve resumen de lo que hace el paquete \texttt{gmapping}.
\item (2 pts) Usando el teclado, mover al robot por todo su entorno para
  generar un mapa del entorno. Al finalizar de generar el mapa, y
  \underline{sin cerrar} el terminal que corre
  \texttt{turtlebot3\_slam\_mapping}, grabar el mapa usando:
\begin{listingshell}
$ rosrun map_server map_saver -f mapa_world1
\end{listingshell}
% $
Adjuntar una captura de pantalla que muestre el mapa obtenido. Luego, mover el
mapa dentro de una carpeta llamada \texttt{autlab4/maps}.

\item (1 pts) Abrir el mapa obtenido como si fuera una imagen (con cualquier
  programa que permita visualizar imágenes) e indicar qué representan los
  valores de los píxeles.

\item (2 pt) Si el robot tuviese como sensor una cámara de profundidad, en
  lugar de un LiDAR, indicar cómo se podría hacer uso del procedimiento
  anterior para la obtención del mapa bidimensional.

\end{enumerate}


\section{Navegación (7 pts)}

Para poder realizar la navegación autónoma de un robot móvil es necesario
primero tener el mapa. En este caso, dado que ya se ha generado el mapa en la
sección anterior, es posible utilizarlo para realizar la navegación. En esta
sección será necesario cargar el mapa, localizar el robot dentro del mapa, y
navegar de un punto inicial a un punto final, considerando la orientación
adecuada. Para esto se utilizará el paquete llamado \texttt{amcl}.

Inicializar nuevamente el simulador de Gazebo para este laboratorio, corriendo
el archivo \texttt{turtlebot3\_world1.launch}. Luego, cargar el mapa
previamente almacenado usando el siguiente comando desde otro terminal:
\begin{listingshell}
$ roslaunch autlab4 turtlebot3_amcl.launch map_file:=camino_a_mapa.yaml
\end{listingshell}
%$
\noindent donde \texttt{camino\_a\_mapa.yaml} es la ruta absoluta al mapa
previamente almacenado. Por ejemplo, podría ser algo como
\texttt{/home/usuario/ros\_ws/mapa\_world1.yaml}. Luego, ejecutar RViz, desde
un terminal diferente, para poder visualizar el entorno usando el siguiente
comando:
\begin{listingshell}
$ roslaunch autlab4 view_navigation.launch --screen
\end{listingshell}
%$

\paragraph{Actividades.}
\begin{enumerate}
\item (2 pts.) Hacer un breve resumen de lo que hace el paquete \texttt{amcl}.
\item (2 pts.) En RViz, hacer click en \textit{2D Pose Estimate} e indicar
  aproximadamente la posición inicial en la que se encuentra el robot. Una vez
  indicada la posición, mover el robot un poco con el teclado para ayudarle a
  su localización. Luego, hacer click en \texttt{2D Nav Goal} y hacer click en
  alguna parte del mapa hasta donde se desea que el robot se mueva. Adjuntar en
  el reporte todo lo mostrado, brindando una explicación objetiva de lo que se
  observa.

\item (2 pts.) Brindar las capturas de pantalla correspondientes a al menos dos
  diferentes posiciones/orientaciones deseadas del robot móvil, donde se
  muestre el camino que se está siguiendo.

\item (1 pt.) ¿Qué sucede con la navegación cuando se añade un objeto nuevo al
  mapa, el cual se interpone en la trayectoria del robot? 
  
\end{enumerate}




\section{Conclusiones (1 pt)}
Mencionar algunas conclusiones relevantes al desarrollo de este laboratorio.


  

\end{document}


