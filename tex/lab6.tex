\documentclass[a4paper,11pt]{robotlabs}

% Makes tables look nicer
\usepackage{booktabs}
\renewcommand{\arraystretch}{1.2}


% To change size of verbatim
\usepackage{verbatimbox}
\include{commands}

\definecolor{lbcolor}{rgb}{0.9,0.9,0.9}
\definecolor{colorkey}{rgb}{0,0,1}
\definecolor{colorcomment}{rgb}{0, 0.6, 0.3}
\definecolor{colorstring}{rgb}{0.627,0.126,0.941}
\definecolor{colornumber}{rgb}{0.205, 0.142, 0.73}
\definecolor{forestgreen}{RGB}{34,139,34}
\definecolor{orangered}{RGB}{239,134,64}
\definecolor{darkblue}{rgb}{0.0,0.0,0.6}
\definecolor{gray}{rgb}{0.4,0.4,0.4}
\definecolor{orangered2}{RGB}{128,0,0}

\newtcblisting{listingshell}
{
  listing only,
  colframe=white,
  listing options={
    basicstyle=\small\ttfamily,
    % columns=full,
    columns=fullflexible,
    literate=
      {\$}{{\textcolor{blue}{\$ }}}{1}
      {\~} {$\sim$}{1}
  },
}

\newtcblisting{listingnodollar}
{
  listing only,
  colframe=white,
  listing options={
    basicstyle=\small\ttfamily,
    % columns=full,
    columns=fullflexible,
    literate=
    %   {\$}{{\textcolor{blue}{\$ }}}{1}
      {\~} {$\sim$}{1}
  },
}

\newtcblisting{listingpython}[1]{%
  listing only,
  colframe=gray!20!white,
  coltitle=black,
  title=\footnotesize{#1},
  listing options={
    language=python,
    upquote=true,
    basicstyle=\small\ttfamily,
    columns=fullflexible,
    extendedchars=false,
    breaklines=true,
    showtabs=false,
    showspaces=false,
    showstringspaces=false,
    identifierstyle=\ttfamily,
    keywordstyle=\color{colorkey},
    commentstyle=\color{colorcomment},
    stringstyle=\color{colorstring},
    numberstyle=\color{colornumber},
    breaklines=True,
    floatplacement=H
  },
}


\newtcblisting{listingxml}[1]{%
  listing only,
  colframe=gray!20!white,
  coltitle=black,
  title=\footnotesize{#1},
  listing options={
    language=XML,
    basicstyle=\small\ttfamily,
    columns=fullflexible,
    extendedchars=false,
    breaklines=true,
    showtabs=false,
    showspaces=false,
    showstringspaces=false,
    identifierstyle=\ttfamily,
    commentstyle=\color{gray},
    keywordstyle=\color{orangered2},
    stringstyle=\color{black},
    tagstyle=\color{colorkey},
    morekeywords={attribute,xmlns,version,type,release,package,name,
      description,maintainer,license,buildtool_depend,build_depend,
      run_depend, length, radius, rpy, xyz, rgba, link, joint,
      effort, velocity, lower, upper},
    otherkeywords={attribute=, xmlns=},
  },
}


\newtcblisting{listingpython2}{%
  listing only,
  colframe=gray!20!white,
  coltitle=black,
  listing options={
    language=python,
    upquote=true,
    basicstyle=\small\ttfamily,
    columns=fullflexible,
    extendedchars=false,
    breaklines=true,
    showtabs=false,
    showspaces=false,
    showstringspaces=false,
    identifierstyle=\ttfamily,
    keywordstyle=\color{colorkey},
    commentstyle=\color{colorcomment},
    stringstyle=\color{colorstring},
    numberstyle=\color{colornumber},
    morekeywords={self}
    breaklines=True,
    floatplacement=H
  },
}

\title{
  \textbf{Laboratorio 6 \\ Localización y Fusión de Datos}
  \date{}
}

\begin{document}

\course{Robótica Autónoma 2020-1}
\period{2020-1}
\labnumber{6}
\labname{Localización y Fusión de Datos}

% Carátula
% -----------
\thispagestyle{empty}
\begin{center}
\textbf{{\huge Universidad de Ingenier\'ia y Tecnolog\'ia}\\ [.5cm]
 {\LARGE Departamento de Ingenier\'ia  Mecatr\'onica}\\[3cm]
}
{\includegraphics[width=6cm]{images/utec}}\\[3cm]
\end{center}

\begin{center}
  {\LARGE \textbf{Robótica Autónoma}}\\[0.8cm]
  {\Large \textbf{Profesor: Oscar Ramos Ponce}}\\[3.0cm]
  {\Large \textbf{Laboratorio 6}}\\[0.5cm]
  {\Large \textbf{Localización y Fusión de Datos}}\\[4.8cm] %[2.0cm]
  {\Large \textbf{Lima - Perú}} \\[0.5cm]
  {\LARGE \textbf{2020-1}}
\end{center}

\newpage
\maketitle
\thispagestyle{fancyplain}

%\tableofcontents
%\thispagestyle{empty}

\section{Objetivos}

\begin{itemize}
\item Aplicar un filtro de Kalman a una unidad de medida inercial (IMU) para
  estimar la posición y orientación de un robot móvil.
\item Predecir el movimiento de un robot móvil diferencial usando el modelo
  basado en entradas de velocidad lineal y angular.
\item Aplicar un filtro de Kalman extendido (EKF) para localizar un robot
  usando mediciones de un IMU y de ARTags detectados mediante una cámara RBG
  montada sobre un robot Turtlebot3 simulado en Gazebo.
\end{itemize}

\section{Equipo y Materiales}

\begin{table*}[h]
  \centering
  \begin{tabular}{cc}
    \toprule
    \textbf{Descripción} & \textbf{Cantidad} \\
    \midrule
    Computadora con conexi\'on a internet & 1 \\
    \bottomrule
    \end{tabular}
  \end{table*}

\section{Aspectos Generales}

Para la presentación del reporte  del laboratorio se debe tener en cuenta lo
siguiente.
\begin{itemize}
\item El reporte debe ser desarrollado utilizando cualquier editor de texto,
  pero debe ser presentado en formato pdf a través del buzón de
  Canvas.
\item En el reporte se debe indicar claramente la sección para la cual se está
  brindando la respuesta a la correspondiente actividad.
\item Se trabajará en grupos de 2 o 3 personas y solo es necesario que un
  integrante de cada grupo presente el reporte, indicando el nombre de los
  integrantes del grupo.
\end{itemize}
Para trabajar en este laboratorio, es necesario actualizar el repositorio de
git clonado en el laboratorio anterior. Para ello, se debe ir a la carpeta
\texttt{autlabs} y se debe ejecutar \texttt{git pull}:
\begin{listingshell}
$ cd ~/lab_ws/src/autlabs
$ git pull
\end{listingshell}
\noindent Una vez realizado esto, dentro de la carpeta \texttt{autlabs} se debe
tener una nueva carpeta llamada \texttt{autlab6}, que es un paquete de ROS y es
donde se trabajará este laboratorio.


\section{Unidad de Medida Inercial (6 pts)}

El robot Turtlebot3 posee algunos sensores propioceptivos y
exteroceptivos. Dentro de los sensores propioceptivos que posee, se encuentra
la unidad de medida inercial (IMU). Para esta sección se puede lanzar el
siguiente entorno de Gazebo:
\begin{listingshell}
$ roslaunch autlab6 turtlebot3_world1.launch
\end{listingshell}
%$
\noindent Este archivo inserta una instancia del robot en el entorno de
Gazebo. Este robot se inicializa con los sensores definidos en su URDF. Así,
los valores del IMU son publicados por el simulador (Gazebo). En un caso real,
estos valores tendrían que ser leidos directamente de un IMU y se requeriría un
nodo que los enviase a un tópico.

En esta sección se realizará la estimación de la posición del robot por
integración numérica, y se generará el \textit{framework} básico que será
utilizado en las siguientes secciones.


\paragraph{Actividades} 

\begin{enumerate}

\item (1 punto) Crear un nodo llamado \texttt{mover\_cuadrado.py} que envíe los
  comandos de velocidad necesarios (al tópico \texttt{/cmd\_vel}) para que el
  robot simulado se mueva aproximadamente en un cuadrado (la longitud del lado
  del cuadrado puede ser escogida arbitrariamente). Adjuntar el código del nodo
  y algunas capturas de pantalla del movimiento en Gazebo.

\item (1 punto) Verificar el URDF del robot Turtlebot3 e indicar dónde se
  encuentra el IMU con respecto al sistema de la base del robot
  (\texttt{base\_link}). Igualmente, indicar dónde se encuentran las ruedas del
  robot con respecto a la base del mismo, y dónde se encuentra el punto medio
  entre ambas ruedas.
  
\item (4 punto) Crear un nodo llamado \texttt{imu\_integr.py} que se suscriba
  al tópico publicado por el IMU (llamado \texttt{/imu}). Este nodo debe
  integrar numéricamente las mediciones de velocidad y aceleración para estimar
  la posición y orientación del robot móvil. El nodo debe, además, publicar la
  posición y orientación estimada con el IMU (de \texttt{base\_footprint} con
  respecto a \texttt{/odom}) como un \texttt{tf}. A esto se suele llamar hacer
  un \textit{broadcast} de un TF. Si se implementa adecuadamente, se debe poder
  visualizar desde RViz el movimiento del robot describiendo aproximadamente un
  cuadrado (se debe escoger \texttt{odom} como \texttt{base\_frame} en
  RViz). Incluir capturas de pantalla, un gráfico de la posición estimada
  (x-y), y comentarios de los resultados.

  \textit{Ayuda:} En la carpeta \texttt{src} hay un ejecutable llamado
  \texttt{tfbroadcast.py} que se puede usar como ejemplo para saber cómo
  realizar el \textit{broadcast} (al ejecutar, en RViz se debe observar una
  posición y orientación arbitraria del robot).

\end{enumerate}  

\section{Estimación de Posición usando el IMU}

En esta sección se hará uso del filtro de Kalman para poder estimar la posición
y orientación del robot, con base en el sensor de medida inercial. Este filtro
requiere de un modelo de movimiento, para la predicción, y de un modelo de
medición, para la actualización o corrección. El modelo de medición estará dado
por las lecturas que realiza el IMU (velocidad angular y aceleración lineal),
mientras que se considerará dos casos de modelo de movimiento.

\subsection{Usando un Modelo de Movimiento Genérico (6 pts)}

Suponiendo que se desconoce por completo el modelo de movimiento que tiene el
robot, o que no se tiene acceso ni a sus comandos de velocidad ni a los
encoders de las ruedas, lo mejor que se puede hacer es estimar un movimiento
genérico de posición y velocidad con base en la expansión de Taylor.

\paragraph{Actividades} 

\begin{enumerate}
\item (5 pts) Crear un nodo llamado \texttt{estimacion1.py} que implemente un
  filtro de Kalman para estimar la posición y orientación del robot móvil. Este
  filtro debe utilizar un modelo de movimiento genérico, y el modelo de
  medición debe incluir las lecturas necesarias del IMU. Se puede escoger
  arbitrariamente la frecuencia a la que se lee el IMU, y se debe escoger
  apropiadamente los estados del filtro de Kalman. Para probar el
  funcionamiento del programa, utilizar el movimiento anterior, en el que el
  robot describe aproximadamente un cuadrado.

  Se debe adjuntar el modelo de movimiento y medición (teóricos), el programa
  desarrollado, una gráfica de la estimación (plano x-y), y comentarios de los
  resultados.

\item (1 pt) Realizar una variación de los valores de las matrices de
  covarianza $R$ y $Q$ en el modelo anterior. Mostrar la gráfica de la
  estimación (plano x-y) y comentar los resultados obtenidos.
  
\end{enumerate}

\subsection{Usando un Modelo de Velocidad (4 pts)}

En esta sección se supone que se conoce los valores de velocidad lineal y de
velocidad angular que se da al robot como señales de control. El conocimiento
de dichos valores permite estimar la posición y orientación del robot a modo de
predicción. 

\paragraph{Actividad (4 pts).} 
Crear un nodo llamado \texttt{estimacion2.py} que implemente un filtro de
Kalman extendido usando como modelo de predicción el modelo de movimiento
basado en velocidad (slide 15 del tema 8). Para los valores de velocidad, el
nodo se puede suscribir al tópico \texttt{/cmd\_vel}. El modelo de medición
será el mismo que el utilizado en la sección anterior. Probar el funcionamiento
con el movimiento en el que el robot describe aproximadamente un cuadrado.

Se debe adjuntar el modelo de movimiento, de medición, así como las
linealizaciones realizadas. Igualmente, se debe adjuntar el programa
desarrollado, una gráfica de la estimación (plano x-y), y comentarios de los
resultados.

\section{Fusión de Datos del IMU y ARTags (4 pts)}

En esta sección se fusionará las mediciones del IMU con las mediciones de
posición de los ARTags, obtenidos a partir de la cámara del robot, para
mejorar la estimación de la posición y orientación del robot. Con este fin, se
seguirá utilizando la trayectoria cuadrada anterior, y se debe localizar al
menos 3 ARTags en el entorno de Gazebo. Los ARTags fueron generados en el
primer laboratorio y se deben encontrar dentro de la carpeta
\texttt{.gazebo}. Según lo realizado en el primer laboratorio, debe haber hasta
18 marcadores: de \texttt{marker0} a \texttt{marker17}. Un ejemplo de cómo
agregar un marcador se puede encontrar en la línea 104 del archivo
\texttt{worlds/turtlebot3\_ar.world}. Para visualizar este archivo, ejecutar:
\begin{listingshell}
$ roslaunch autlab6 turtlebot3_aarworld.launch
\end{listingshell}
%$
\noindent Se debe observar un ARTag (\textit{marker} número 0) en la posición
$(2,1)$, con una orientación de 90 grados. Como se muestra en el archivo
indicado, se puede añadir más marcadores (tienen que ser diferentes, para su
fácil detección). La posición de cada marcador se describe dentro del tag
llamado \texttt{pose} y es $x, y, z$ (donde $z=0$ ya que se está en el plano)
seguido de la orientación $r_x, r_y, r_z$ (donde $r_z$ es la rotación alrededor del
eje $z$, y los otros dos valores deben ser cero ya que se está en el plano
$x,y$). 

\paragraph{Actividad (4 pts).} 
Crear un nodo llamado \texttt{estimacion3.py} que implemente un filtro de
Kalman extendido usando como modelo de predicción el modelo de movimiento
basado en velocidad, al igual que en el caso anterior. El modelo de movimiento
debe incluir las lecturas del IMU y las lecturas de los ARTags, realizados con
la cámara. La frecuencia con la que se lee ambos sensores puede ser diferente,
y se tiene que tener en cuenta que para la actualización (o corrección) del
filtro solo se toma la medición realizada en dicho instante de tiempo (no se
almacena mediciones). La lectura de los ARTags debe realizarse como en el
primer laboratorio. Probar el funcionamiento con el movimiento en el que el
robot describe aproximadamente un cuadrado.

Se debe adjuntar el modelo de movimiento, de medición, así como las
linealizaciones realizadas. Igualmente, se debe adjuntar el programa
desarrollado, una gráfica de la estimación (plano $x$-$y$), y comentarios de los
resultados.

  

\section*{Nota}

Si bien no es necesario para este laboratorio la visualización de la matriz de
covarianza como elipse, esta vizualización puede ser útil en muchos casos para
saber visualmente cuán seguro se está sobre la posición y orientación del
robot. Para visualizar esta elipse en RViz, se puede utilizar el paquete
llamado \texttt{rviz\_plugin\_covariance}, el cual puede ser clonado en el
repositorio \texttt{https://github.com/laas/rviz\_plugin\_covariance}. Una vez
clonado, se debe compilar ya que contiene código de C. Para visualizar, en RViz
se debe añadir un elemento llamado \textit{PoseWithCovariance}. Luego, se puede
ejecutar, por ejemplo \texttt{send\_covariance\_msgs\_2D.py}, que se encuentra
en la carpeta \texttt{scripts} del repositorio clonado, para visualizar la
elipse y una flecha que indica la orientación. Modificando adecuadamente este
nodo se puede añadir la visualización de la covarianza para las posiciones y
orientaciones del robot.



\end{document}



%%% Local Variables:
%%% mode: latex
%%% TeX-master: "lab6"
%%% End:
